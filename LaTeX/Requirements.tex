\documentclass[10pt]{article}
\usepackage{graphicx} %Required for diagrams
\usepackage{bookmark}
\usepackage{hyperref}


\begin{document}

\begin{titlepage}
\begin{center}
	\begin{figure}[t]
		\centering
		\includegraphics[width=350px]{UP_Logo.png}
	\end{figure}
	
\textsc{\LARGE COS301 Mini Project \newline\newline Architectural Requirements Specification}
%\begin{minipage}{0.4\textwidth}
		\textbf{\newline Group 5a} \\
		\begin{flushright} \large
			Matthew Gouws \emph{u11008602} \newline
			Tsepo Ntsaba \emph{u10668544} \newline
			Werner Mostert \emph{u13019695} \newline
			Semaka Malapane \emph{u13081129} \newline
			Name Surname \emph{uXXXXXXXX} \newline
			Name Surname \emph{uXXXXXXXX} \newline
			Name Surname \emph{uXXXXXXXX} \newline
			Name Surname \emph{uXXXXXXXX} \newline
		\end{flushright}
		%\end{minipage}
		
		\vfill
		
		{\large Version }
		\\
		{\large \today}
		
\end{center}
\end{titlepage}

\newpage
\tableofcontents
\newpage

\section{Introduction}
The Buzz system is to be developed to enhance user interaction by means of a online discussion board, this board will have multiple components to allow for better user interactions by means of rewards, ranks and privileges. This document serves as the Architectural specifications for the Buzz system. Including the constraints imposed by the client Ms. Vrede Pieterse.

\section{Vision}
This project should help to engage students and encourage learning amongst students. The system should be able to be integrated into any website to further help other universities with the same results.

\section{Background}
\begin{itemize}
\item The system is to be developed to help the CSEDAR (Computer Science Education Didactic and Applications Research.
\item Improve on the current discussion board already in place for the University of Pretoria Department of computer science.
\item Create a collaborative group of students to help solve problems before escalating to a higher knowledge source.
\end{itemize}


\section{Architectural Requirements}
\subsection{Access channel and integration requirements} 
\subsection{Introduction}
The BuzzSpace system is designed to be an interactive platform in which students within the same course, module and/or department can contribute and assist each other. Given the number of people using the system, not all of them will have the same amount of access to Internet. With this in mind, the system must be designed to accommodate such situation.

\subsection{Web Platforms}
The BuzzSpaces system is primarily designed to be accessed through web browser applications such as Firefox and Safari. The primary design of the architecture should be to handle user making use of such applications. The design should accommodate for different browser platforms as not everyone uses the same ones.


\subsection{Mobile Platform}
Web browsers have also been ported to mobile phones. This allows for users to access the web version of the BuzzSpace through their mobile phones. Given the fact that they are mobile, phones, not all the of the functionality that a full website provides will be abailable for mobile web users.

Mobile applications for iOS, Android and Windows phones have grown in popularity. These applications can designed to run alongside the main website and provide the same functionality relative to the device. 

\paragraph{System access channels}
\subsubsection{Integration Channels}
\paragraph{Integration Channel}
\begin{itemize}
\item \hfill The Enterprise service bus will be used. The bus is a subsystem that facilitates the communication between other subsystems. This means that subsystems only need to communicate through the ESB. This significantly lowers the amount of necessary connections and it processes the messages, meaning other subsystems need not waste time processing messages. If a subsystem is replaced or added, the only changes that need to take place is the creation of a new interface between the new system and the bus. This makes the system much more flexible  and makes communication much more simple.
\end{itemize}

\paragraph{Protocols Used}
\begin{itemize}
\item Hypertext Transfer Protocol

HTTP is the standard way and method to transfer method of web pages over the internet by using the protocols TCP/IP (Transmission Control Protocol and Internet Protocol) to manage the Web transmission. In use of Buzz all posting, registering and even just access to the forums will make use of these protocols.

\item Mail Protocols POP3 and SMTP

Post Office Protocol (POP) will probably not be used within the system but is used in receiving emails. Simple Mail Transfer Protocol (SMTP) will be used though, the protocol is used in the sending of emails by making use of a SMTP server. In the case of Buzz all notifications that are not present within Buzz will be sent through email such as account verification and notification of changes within a user’s account. Finally if a user loses account details all information and lost password would be sent over email.

\item File Transfer Protocol

File Transfer Protocol (FTP) is the method of copying files over the internet or a network but more simply put allows for the transfer of information from one computer to a remote computer. In terms of Buzz we are not creating a File Server but rather with all the images profile pictures and any other documentation that are uploaded to different profiles for other users to use make use of the File Transfer Protocol.

\end{itemize}




\clearpage

\subsection{Architectural responsibilities}  % Matthew
\begin{itemize}
	\item The System must be able to provide concurrent clients to read threads, post threads and update threads.
	\item The system should be able to store all threads and posts, as well as who posted, and whether a thread has been deleted.
	\item The system should provide an integration environment to allow for multiple deployment
	\item The system could allow for persistent data storage for easy 'Remember Me'
	\item Storage of archived thread
	
\end{itemize}
\clearpage
\subsection{Quality requirements} % Godfrey
\subsubsection{Scalability}

The  Buzz system , when neccesary or needed  must be able to be easily converted into an application to be used on mobile phones. The system should be built using independent components (MVC pattern)  to ensure separition of concerns,cohesion,decoupling of components and pluggability when developing it for mobile.

\subsubsection{Performance requirements}

The system needs to respond almost instantly in terms of creation and deletion of a Buzz space or creation and deletion of threads as well as the management or administration of the whole system and it must also be able to handle high contention where a lot of concurrent users will want to access an actice buzz space at once where they should not experience latency or be susceptible to data traffic.

\subsubsection{Maintainabilty}

The system should be able to be continually improved by developers who worked on it initially and new developers.The system should also have the attribute that it can be restored.

\subsubsection{Reliability and Availability}

The buzz system needs to be 'online' almost every time.The server or host for the system should then be able to be used by users at any time. Also , threads or post must be recent and for the administrators the system must produce correct results when queried.


\subsubsection{Security}

The system as a whole should be able to protect the information of students and lectures from external infiltration e.g possible SQL injection. Again, it should provide some form of validation mechanism to ensure that only students that are allowed on specific Buzz spaces are allowed on a certain one. Also , the mechanism needs to block out all students who are not allowed on the system.

\subsubsection{Monitorability and Auditability}

Buzz should provide a way such that it can be audited or viewed at particular check points  i.e it can keep track of the number
of posts, keep track of dates and be able to link a post to a student. This will require to keep a record of its states during different days or months and with this we can employ the Memento design pattern.

\subsubsection{Maintainability}

The system should be able to be continually improved by developers who worked on it initially and new developers.
The system should also have the attribute that it can be restored.

\subsubsection{Usability}

The system must be user friendly and easy to use for users. Users must be able to interact with the system i.e the user interface must be easy to learn and understand.

\subsubsection{Integrability}

It should be easy to combine The Buzz system with other modules or assistance technologies like a Database Management System.It should be able to export data into various other modules.


\clearpage

\subsection{Architecture constraints} % Semaka & Werner

\begin{description}
		\item[Reference Architecture] \hfill \\
		 	 \begin{itemize}	
				\item Java EE (Enterprise Edition) is specified as the chosen Reference Architecture to use.
					\begin{description}
						\item[Details] \hfill \\
							Java Platform, Enterprise Edition or Java EE is Oracle's enterprise Java computing platform. The platform provides an
							 API and runtime environment for developing and running enterprise software, including network and web services, and
							 other large-scale, multi-tiered, scalable, reliable, and secure network applications. Java EE extends the Java Platform, 
							Standard Edition (Java SE),  providing an API for object-relational mapping, distributed and multi-tier architectures, and
							 web services. 
						\item[Comments]\hfill \\		
					\end{description}
			\end{itemize}
		\item[Other Software Technologies] \hfill \\
			\begin{itemize}
				\item JPA (Java Persistence API)
					\begin{description}
						\item[Details] \hfill \\
							The Java Persistence API (JPA) is a Java programming language application programming interface specification that
							 describes the management of relational data in applications using Java Platform, Standard Edition and Java Platform, Enterprise Edition.
						\item[Comments]\hfill \\
							
					\end{description}
				\item JPQL (Java Persistence Query Language)
					\begin{description}
						\item[Details] \hfill \\
							The Java Persistence Query Language (JPQL) is a platform-independent object-oriented query language defined as part
							 of the Java Persistence API (JPA) specification. JPQL is used to make queries against entities stored in a relational database.
							 It is heavily inspired by SQL, and its queries resemble SQL queries in syntax, but operate against JPA entity objects rather
							 than directly with database tables.
						\item[Comments]\hfill \\
							
					\end{description}
				\item JSF (JavaServer Faces)
					\begin{description}
						\item[Details] \hfill \\
							JavaServer Faces (JSF) is a Java specification for building component-based user interfaces for web applications and
							 exposing them as server side Polyfills.] It was formalized as a standard through the Java Community Process and is
							 part of the Java Platform, Enterprise Edition.
						\item[Comments]\hfill \\	
					\end{description}
				\item HTML (HyperText Markup Language)
					\begin{description}
						\item[Details] \hfill \\
							HyperText Markup Language, commonly referred to as HTML, is the standard markup language used to create web pages.
						\item[Comments]\hfill \\
							
					\end{description}
				\item AJAX (Asynchronous JavaScript and XML)
					\begin{description}
						\item[Details] \hfill \\
							Ajax is a group of interrelated Web development techniques used on the client-side to create asynchronous Web applications.
							 With Ajax, web applications can send data to and retrieve from a server asynchronously (in the background) without interfering
							 with the display and behavior of the existing page.
						\item[Comments]\hfill \\		
					\end{description}
				\item CSS
					\begin{description}
						\item[Details] \hfill \\
							Cascading Style Sheet will be used for the formatting and styling of the discussion forum. It will ensure separation of the content from presentation.
						\item[Comments]\hfill \\
					\end{description}	
				\item Github
					\begin{description}
						\item[Details] \hfill \\
							Github will be the platform used by the developers to collaborate. This will be where all the code will be pushed so that the developers can be able to work together from remote locations and be able to combine all their code/work.
						\item[Comments]\hfill \\
					\end{description}
			\end{itemize}	
		\item[Operating System] \hfill 
			\begin{itemize}
				\item Linux
					\begin{description}
						\item[Details] \hfill \\
							The forum will  be designed to work on any Operating system. However, Linux is ideal for certain functionalities therefore the forum will work optimally on Linux Operating Systems. The web server will be hosted on a Linux machine as Linux is widely considered to be the best operating system  for web servers. 
						\item[Comments]\hfill \\
							
					\end{description}
			\end{itemize}	
		\item[Deployed Environments] \hfill 
			\begin{itemize}
				\item Web Based Interface
					\begin{description}
						\item[Details] \hfill \\
							Users will be able to navigate to the application using a web browser from any compatible device, such as a tablet; smartphone or computer.
						\item[Comments]\hfill \\
							This is a very suitable environment since it allows for a large measure of ease of access.
					\end{description}
				\item Android (out of scope - for future reference)

			\end{itemize}
\end{description}	

\end{document}
