\documentclass[10pt]{article}
\usepackage{graphicx} %Required for diagrams
\usepackage{bookmark}
\usepackage{hyperref}

\begin{document}

\begin{titlepage}
\begin{center}
	\begin{figure}[t]
		\centering
		\includegraphics[width=350px]{UP_Logo.png}
	\end{figure}
	
\textsc{\LARGE COS301 Mini Project \newline\newline Architectural Requirements Specification}
%\begin{minipage}{0.4\textwidth}
		\textbf{\newline Group 5a} \\
		\begin{flushright} \large
			Matthew Gouws \emph{u11008602} \newline
			Tsepo Ntsaba \emph{u10668544} \newline
			Werner Mostert \emph{u13019695} \newline
			Semaka Malapane \emph{u13081129} \newline
			Name Surname \emph{uXXXXXXXX} \newline
			Name Surname \emph{uXXXXXXXX} \newline
			Name Surname \emph{uXXXXXXXX} \newline
			Name Surname \emph{uXXXXXXXX} \newline
		\end{flushright}
		%\end{minipage}
		
		\vfill
		
		{\large Version }
		\\
		{\large \today}
		
\end{center}
\end{titlepage}

\newpage
\tableofcontents
\newpage
\listoffigures
\newpage

\section{Introduction}
text goes here

\section{Vision}
text goes here

\section{Background}
text goes here

\section{Architectural Requirements}
\subsection{Access channel and integration requirements} 
\subsubsection{Access channels}
\paragraph{Human access channels}
\paragraph{System access channels}
\subsubsection{Integration Channels}
\clearpage

\subsection{Architectural responsibilities}  % Matthew
\begin{itemize}
	\item The System must be able to provide concurrent clients to read threads, post threads and update threads.
	\item The system should be able to store all threads and posts, as well as who posted, and whether a thread has been deleted.
	\item The system should provide an integration environment to allow for multiple deployment
	\item The system could allow for persistent data storage for easy 'Remember Me'
	\item Storage of archived thread
	
\end{itemize}
\clearpage
\subsection{Quality requirements} % Godfrey
\subsubsection{Scalability}

\subsubsection{Performance requirements}

\subsubsection{Maintainabilty}

\subsubsection{Reliability and Availability}

\subsubsection{Security}

\subsubsection{Monitorability and Auditability}

\subsubsection{Testability}

\subsubsection{Usability}

\subsubsection{Integrability}


\clearpage

\subsection{Architecture constraints} % Semaka & Werner

\begin{description}
		\item[Reference Architecture] \hfill \\
		 	 \begin{itemize}	
				\item Java EE (Enterprise Edition) is specified as the chosen Reference Architecture to use.
					\begin{description}
						\item[Details] \hfill \\
							Java Platform, Enterprise Edition or Java EE is Oracle's enterprise Java computing platform. The platform provides an
							 API and runtime environment for developing and running enterprise software, including network and web services, and
							 other large-scale, multi-tiered, scalable, reliable, and secure network applications. Java EE extends the Java Platform, 
							Standard Edition (Java SE),  providing an API for object-relational mapping, distributed and multi-tier architectures, and
							 web services. 
						\item[Comments]\hfill \\		
					\end{description}
			\end{itemize}
		\item[Other Software Technologies] \hfill \\
			\begin{itemize}
				\item JPA (Java Persistence API)
					\begin{description}
						\item[Details] \hfill \\
							The Java Persistence API (JPA) is a Java programming language application programming interface specification that
							 describes the management of relational data in applications using Java Platform, Standard Edition and Java Platform, Enterprise Edition.
						\item[Comments]\hfill \\
							
					\end{description}
				\item JPQL (Java Persistence Query Language)
					\begin{description}
						\item[Details] \hfill \\
							The Java Persistence Query Language (JPQL) is a platform-independent object-oriented query language defined as part
							 of the Java Persistence API (JPA) specification. JPQL is used to make queries against entities stored in a relational database.
							 It is heavily inspired by SQL, and its queries resemble SQL queries in syntax, but operate against JPA entity objects rather
							 than directly with database tables.
						\item[Comments]\hfill \\
							
					\end{description}
				\item JSF (JavaServer Faces)
					\begin{description}
						\item[Details] \hfill \\
							JavaServer Faces (JSF) is a Java specification for building component-based user interfaces for web applications and
							 exposing them as server side Polyfills.] It was formalized as a standard through the Java Community Process and is
							 part of the Java Platform, Enterprise Edition.
						\item[Comments]\hfill \\	
					\end{description}
				\item HTML (HyperText Markup Language)
					\begin{description}
						\item[Details] \hfill \\
							HyperText Markup Language, commonly referred to as HTML, is the standard markup language used to create web pages.
						\item[Comments]\hfill \\
							
					\end{description}
				\item AJAX (Asynchronous JavaScript and XML)
					\begin{description}
						\item[Details] \hfill \\
							Ajax is a group of interrelated Web development techniques used on the client-side to create asynchronous Web applications.
							 With Ajax, web applications can send data to and retrieve from a server asynchronously (in the background) without interfering
							 with the display and behavior of the existing page.
						\item[Comments]\hfill \\		
					\end{description}
				\item CSS
					\begin{description}
						\item[Details] \hfill \\
							Cascading Style Sheet will be used for the formatting and styling of the discussion forum. It will ensure separation of the content from presentation.
						\item[Comments]\hfill \\
					\end{description}	
				\item Github
					\begin{description}
						\item[Details] \hfill \\
							Github will be the platform used by the developers to collaborate. This will be where all the code will be pushed so that the developers can be able to work together from remote locations and be able to combine all their code/work.
						\item[Comments]\hfill \\
					\end{description}
			\end{itemize}	
		\item[Operating System] \hfill 
			\begin{itemize}
				\item Linux
					\begin{description}
						\item[Details] \hfill \\
							The forum will  be designed to work on any Operating system. However, Linux is ideal for certain functionalities therefore the forum will work optimally on Linux Operating Systems. The web server will be hosted on a Linux machine as Linux is widely considered to be the best operating system  for web servers. 
						\item[Comments]\hfill \\
							
					\end{description}
			\end{itemize}	
		\item[Deployed Environments] \hfill 
			\begin{itemize}
				\item Web Based Interface
					\begin{description}
						\item[Details] \hfill \\
							Users will be able to navigate to the application using a web browser from any compatible device, such as a tablet; smartphone or computer.
						\item[Comments]\hfill \\
							This is a very suitable environment since it allows for a large measure of ease of access.
					\end{description}
				\item Android (out of scope - for future reference)

			\end{itemize}
\end{description}	

\end{document}